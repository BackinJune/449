%%%%%%%%%%%%%%%%%%%%%%%%%%%%%%%%%%%%
%% Template file SP 2024
%% Include in directory homework.sty and headerfooter.tex
%%%%%%%%%%%%%%%%%%%%%%%%%%%%%%%%%%%%

\documentclass[12pt]{article}
\usepackage{homework}

\graphicspath{{images/}}
\geometry{letterpaper, portrait, includeheadfoot=true, hmargin=1in, vmargin=1in}

\setcounter{section}{-1}
%% Solution hiding %%
\usepackage[utf8]{inputenc}
\usepackage{lipsum}


\begin{document}
\singlespacing

\renewcommand{\familydefault}{\rmdefault}
\pagestyle{fancy}
\fancyhf{}
\setlength{\headheight}{30pt}
\renewcommand{\headrulewidth}{0.4pt}
\renewcommand{\footrulewidth}{0.4pt}
\lhead{\large Homework 2 \\ Due Feb. 20, 2024 }
\rhead{\large CS 446 \\ Spring 2024}
\rfoot{\textbf{Page \thepage}}
\lfoot{}

\section{Instructions}

Homework is due Tuesday, March 18, 2024 at 23:59pm Central Time.
Please refer to \url{https://courses.grainger.illinois.edu/cs446/sp2024/homework/hw/index.html} for course policy on homeworks and submission instructions.

\section{Neural networks for simple functions}
\begin{enumerate}
    \item   \[w_0 = w_1 = [1, -1]^{\top}\]
    \item   \[w_0 = [1, -1]^{\top}, \, w_1 = [m, -m]^{\top}, \, b = [\frac{b}{m}, -\frac{b}{m}]^{\top}\]
    \item   \[w_0 = 0, \, w_1 = 2b\]
    \item   \[w_0 = [1, 1, 1]^{\top}, \, b_0 = [2, 0,-2]^{\top}, \, w_1 = [3, -6, 3]^{\top}\]
    \item   No. $x^2 = |x|^2$, while any transformation in $f(x)$ is either making linear transforms ($w$), or resulting in 0 ($\sigma$), in which $0 \leq |x|$. Any combination of them is impossible to produce a quadratic transform.
    \item   
    \item 
\end{enumerate}
\newpage

\section{Backpropagation through time (BPTT)}
\begin{enumerate}
    \item \begin{figure}[!htb]
        
    \end{figure}
    \item   \[h_0 = 0, \, h_1 = w(x_1 + h_0) = wx_1, \, h_2 = w(x_2 + h_1) = wx_2 + w^2x_1,\] 
    \[h_3 = w(x_3 + h_2) = wx_3 + w^2x_2 + w^3x_1\]
    \[y_1 = x_1 + h_0 = x_1, \, y_2 = x_2 + h_1 = x_2 + wx_1, \, y_3 = x_3 + h_2 = x_3 + wx_2 + w^2x_1\]
    \item    \[\frac{\partial h_3}{\partial w} = x_3 + (h_2 + w\frac{\partial h_2}{\partial w}) = x_3 + wx_2 + w^2x_1 + w[x_2 + (h_1 + w\frac{\partial h_1}{\partial w})]\] 
    \[= x_3 + 2wx_2 + 2w^2x_1 + w^2(\frac{\partial h_1}{\partial w}) = x_3 + 2wx_2 + 3w^2x_1\]
    \item    \[\frac{\partial f}{\partial h_1} = \frac{\partial h_T}{\partial h_1} = \]
    \item    \[\]
\end{enumerate}
\newpage

\section{Transformers}
\begin{enumerate}
    \item Theoretially $\alpha_i$ can't be infinitely large since attention weights are suuposed to be designed between 0 and 1. Practically the numerator in $\alpha_i$ would be less then denominator, and both of them can't reach 0 due to exponentials. In this way, $\alpha_i$ can neither be neither infinitely large nor be 0.
    \item $q$ and $k_i$ are pointing to similiar directions, which differs from the rest $k$. $c = \sum \alpha_i v_i$ for all $i$ satisfiying such condition. This means that the token corresponding to $q$ here is attached with greater attention to tokens corresponding to such $i$.
    \item 
    \item 
    \item 
\end{enumerate}
\newpage

\section{Resnet}
\begin{enumerate}
    \item[2.] 
\end{enumerate}
\newpage

\section{Coding: Image overfitting}
\begin{enumerate}
    \item[5.]   
    ReLU:
    \begin{enumerate}
        \item epochs = 500
        \item learning rate = $10^{-4}$
        \item batch size = 800
    \end{enumerate}
    Tanh:
    \begin{enumerate}
        \item epochs = 500
        \item learning rate = $10^{-4}$
        \item batch size = 800
    \end{enumerate}
    Sin:
    \begin{enumerate}
        \item epochs = 40
        \item learning rate = $10^{-4}$
        \item batch size = 800
    \end{enumerate}

    \item[6.] 
    \item[7.] 
    \item[8.] 
        
\end{enumerate}
% Resnet code should be submitted on gradescope
% Include plots and results from Q5.5 and written responses for 5.6-8

\end{document}